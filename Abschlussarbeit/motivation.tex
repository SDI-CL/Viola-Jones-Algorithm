Autonomes Fahren ist eines der großen Themen der Automobilindustrie der letzten Jahre. Der Traum vom eigenständig fahrenden Auto besteht schon länger, jedoch gestaltet sich die komplett selbstverwaltende Fahrzeugsteuerung als schwierig, da unheimlich viele Faktoren miteinbezogen sowie analysiert werden müssen.\\
Einer dieser Faktoren ist die erfolgreiche und korrekte Erkennung von Straßenschildern sowie wegweisenden Objekten, in diesem Fall Verkehrshütchen. Ansätze hierzu gibt es viele, jedoch ist die Forschung der Automobilindustrie meist geheim, weshalb eigene Ansätze kreiert werden müssen.\\
Dieses Projekt stellt den Versuch da, sich diesem Thema mit einer Methode zu nähern. Das Projekt verfolgte die Verwirklichung von Zwei Anforderungen:
\begin{enumerate}
\item Es sollen zwei nebeneinanderstehende Verkehrshütchen, auch Pylonen genannt, korrekt erkannt werden und die Position der erkannten Objekte im Bild zurückgegeben werden.
\item 3 Arten von Verkehrszeichen, das Stopschild, Vorfahrt gewähren sowie das Vorfahrtszeichen sollen erkannt werden.
\end{enumerate}
Das Ziel, was hierbei verfolgt wird ist, dass einerseits die Erkennung von Pylonen dazu genutzt werden kann einen Parcours aufzustellen, durch den das Auto eigenständig navigiert, sowie die Erkennung von Verkehrszeichen auf einer Teststrecke für korrektes Fahrverhalten genutzt werden kann. Die Voraussetzung ist, dass dazu der Algorithmus von Viola und Jones , welcher in einem Paper im Jahr 2001 vorgestellt wurde, für diesen Anweundgsfall adaptiert und genutzt wird.