Die Idee zur Umsetzung des Projekts, nämlich den Viola-Jones Algorithmus 
für Gesichtserkennung umzufunktionieren und Schilder beziehungsweise 
Pylonen damit zu erkennen, ließ sich nicht ohne weiteres durchführen.
Es mussten gewisse Vorbereitungen getroffen werden um etwaige Probleme 
zu meistern. Hierzu zählte unter anderem eine Version des 
Viola-Jones mit Python unter ROS zum laufen zu bringen.
\par
Aufgrund der speziellen Architektur des Rapsberry Pi Modell 3, auf dem 
gearbeitet wurde, ließ sich dieses Problem nicht sofort bewerkstelligen. 
Wir liefen in eine Vielzahl von Problemen beim Compilingprozess:
\begin{itemize}
\item[Überhitzung] Der Compileprozess dauerte lange. Währenddessen lief 
der Rasberry unter Hochtouren, was eine Überhitzung zur Folge hatte. Den 
Stecker zu ziehen war daraufhin die einzige Möglichkeit, den Raspberry 
wieder in den Griff zu bekommen. Es musste folglich eine alternative 
Version von OpenCV gesucht werden, die wir mit Python kompilieren 
konnten.
\item[Auslagerungsdatei] Die Auslagerungsdatei des Betriebssystems war 
zuerst zu klein eingestellt. Dies hatte zur Folge, dass an einem 
gewissen Punkt des Kompilierprozesses dieser einfach abbrach, mit einer 
kryptischen Fehlermeldung. Wir erweiterten die Auslagerungsdatei 
anschließend um dieses Problem zu beseitigen.
\item[OpenCV] Für den Raspberry Pi gab es kein fertiges OpenCV Package 
zum herunterladen, weshalb wir dieses selbst bauen mussten. Für die von 
uns verwendet Version waren einige Packages nötig, die wir 
nachinstallieren mussten. Jedoch wurden uns die Abhängigkeiten erst 
während des Kompilierprozesses mitgeteilt, weshalb wir diesen mehrmals 
neu starten mussten.
\item[Korrekte Python Version] Die aktuellste installierte Python 
Version auf dem ROS ist 3.0. Um OpenCV jedoch kompilieren zu können 
wurde Python in der Version 2.7 gebraucht. Dies stellte ein großes 
Problem dar, da eine ganz bestimmte Version von Python verwendet werden 
musste, und es schwer war, genau die richtige zu selektieren und 
anschließend diese im Kompilierprozess zu verwenden.
\end{itemize}
Nachdem wir diese Probleme bewältigt hatten, konnten wir mit dem 
eigentlichen Teil des Projektes beginnen.
