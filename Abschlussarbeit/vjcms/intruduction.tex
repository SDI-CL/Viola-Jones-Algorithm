Dieser Abschnitt des Berichts befasst sich mit der Implementierung unserer Software zur Analyse der Bilddaten des Raspberry Pi's, im weiteren Pi abgekürzt. Der Name VJCMS (Viola-Jones Camshift) kommt von einem Programm, dass wir zu beginn unserer Recherechen gefunden haben.\cite{vjcms} Dieses Programm bzw. Skript, da es sich um ein Python-Skript handelt, bildet funktional die Grundlage unserer Implementierung. Im Verlaufe der Arbeit haben wir das Konzept des Programms aufgenommen und in einer für unsere Zwecke dienlichen Form implementiert.
In seiner ursprünglichen Form hat das Skript nicht nur Gesichter erkannt, sondern diese auch für eine vorher definierte Zeit mittels Camshift Algorithmus verfolgt. 

 