Nach lang anhaltenden Schwierigkeiten mit dem Traingsprozess, der auf lückenhafte Dokumentation sowie nicht vorhandene Erläuterungen zur Benutzung der durch OpenCV bereitgestellten Tools zurückzuführen ist, gelang es uns für die Verkehrsschilder solide Klassifizierer zu erstellen. Die Dauer des Trainingsprozesses nahm von Objekt zu Objekt stark ab. Dauerte es noch mehrere Monate den ersten funktionierenden Klassifizierer für Pylonen zu erstellen, gelang es uns den Klassifizierer für das Vorfahrt gewähren Schild binnen weniger Stunden, inklusive der Trainingszeit für den Algorithmus, zu erstellen.\\ 
Die trainierten Klassifizierer erkennen die Verkehrsschilder mit einer sehr hohen Trefferrate aus verschiedenen Sichtwinkeln. Einziger Negativpunkt hierbei ist die auf die Kamera einfallende Lichtintensität. Ist diese zu hoch, stößt der Klassifizierer schnell an seine Grenzen. Dunkel belichteter Videoinput scheint für unsere Klassifizierer kein Problem darzustellen. Ein Grund weshalb die Verkehrsschilder gut erkannt werden können ist ihre Form. Die genutzten Weak Classifier passen gut auf die Kanten, welche bei allen drei von uns angelernten Verkehrsschildern vorkommen. Die 45\textdegree {} Kanten werden durch die Erweiterung des Modus beim Trainingsprozess als weiteres Indiz zum Erkennen genutzt, weshalb eine sehr hohe Trefferrate mit wenigen Falsepositives erreicht werden konnte.\\
Einzig die Pylonen stellen eine Besonderheit dar. Aufgrund ihrer ungewöhnlichen Form, die durch die Kanten im 70\textdegree {} Winkel zustande kommt, sowie ihrer Art der Verzerrung sollten sie von der Seite betrachtet werden, konnte kein Klassifizierer von uns erstellt werden, der Objekte eindeutig identifizieren konnte. Das Das Muster, welches vorwiegend durch den Viola-Jones Algorithmus erkannt wird, ist hierbei durch die Farbwechsel von rot zu weiß und andersherum, beziehungsweise in Graustufen auf Ebene des Algorithmus geprägt. Diese Art Muster kann je nach Belichtungsverhältnis häufig und leicht in anderen Oberflächen gefunden werden, weshalb eine sichere Nichterkennung von Falsepositives nicht gewährleistet kann. Ein weiterer Grund für potenzielle Falsepositives ist, dass die Weak Classifier nicht in ihrer vollen Funktion zum tragen kommen. Wie bereits oben erwähnt besitzen Pylonen eine steile Kante von circa 70\textdegree . Schwache Klassifizierer sind jedoch für 0\textdegree, 45\textdegree {} und 90\textdegree {} angepasst, weshalb diese im Lernprozess kaum zum Tragen kommen. Die Erkennung von Pylonen beruht somit alleinig auf dem rot-weiß Muster, welches das Charakteristikum der Eindeutigkeit gegenüber anderen Objekten, die im Alltag vorkommen, leider nicht vorweist.\\
Zusammenfassend kann gesagt werden, dass der Algorithmus nach einiger Einarbeitungszeit eine solide Basis zur Verkehrsschildererkennung darstellt und für Pylonen nur bedingt geeignet ist. Um die Erkennung von Pylonen sinnvoll zu nutzen wird vorgeschlagen weitere logische Bedingungen einzufügen, sodass ein sinnvolles Geleit durch die Pylonenpaare gewährleistet werden kann. Die Umsetzung auf dem Pi ist als erster Versuch und somit Übergangslösung anzusehen, da die gemessene Performance auf diesem Gerät nicht überzeugt.
Um diese verbessern zu können gibt es zweierlei Ansätze. Zum einen besteht die Möglichkeit eine Hochsprache für die Implementierung zu verwenden, die ein besseres Resourcenmanagement erlaubt und insgesamt performanter ausgeführt wird. Zum anderen ist eine leistungsfähigere Hardware einzusetzen. Empfehlenswert wäre an dieser Stelle die Implemenierung einzelner Schritte bzw. Berechnungen auf dem Z-Board, um besonders die rechenintensive Erkennung von Objekten zu beschleunigen. Damit sollte dann eine deutlich bessere Performance und somit ein höhere FPS Rate zu erreichen sein.